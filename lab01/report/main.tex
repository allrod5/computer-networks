%%%%%%%%%%%%%%%%%%%%%%%%%%%%%%%%%%%%%%%%%
% Programming/Coding Assignment
% LaTeX Template
%
% This template has been downloaded from:
% http://www.latextemplates.com
%
% Original author:
% Ted Pavlic (http://www.tedpavlic.com)
%
% Note:
% The \lipsum[#] commands throughout this template generate dummy text
% to fill the template out. These commands should all be removed when 
% writing assignment content.
%
% This template uses a Perl script as an example snippet of code, most other
% languages are also usable. Configure them in the "CODE INCLUSION 
% CONFIGURATION" section.
%
%%%%%%%%%%%%%%%%%%%%%%%%%%%%%%%%%%%%%%%%%

%----------------------------------------------------------------------------------------
%	PACKAGES AND OTHER DOCUMENT CONFIGURATIONS
%----------------------------------------------------------------------------------------

\documentclass{article}

\usepackage[utf8]{inputenc}
\usepackage[brazil]{babel}
\usepackage{fancyhdr} % Required for custom headers
\usepackage{lastpage} % Required to determine the last page for the footer
\usepackage{extramarks} % Required for headers and footers
\usepackage[usenames,dvipsnames]{color} % Required for custom colors
\usepackage{graphicx} % Required to insert images
\usepackage{listings} % Required for insertion of code
\usepackage{courier} % Required for the courier font
\usepackage{lipsum} % Used for inserting dummy 'Lorem ipsum' text into the template

% Margins
\topmargin=-0.45in
\evensidemargin=0in
\oddsidemargin=0in
\textwidth=6.5in
\textheight=9.0in
\headsep=0.25in

\linespread{1.1} % Line spacing

% Set up the header and footer
\pagestyle{fancy}
\lhead{\hmwkAuthorName} % Top left header
\rhead{\hmwkClass\ (\hmwkClassInstructor\ \hmwkClassTime): \hmwkTitle} % Top center head
%\rhead{\firstxmark} % Top right header
\lfoot{\lastxmark} % Bottom left footer
\cfoot{} % Bottom center footer
\rfoot{Page\ \thepage\ of\ \protect\pageref{LastPage}} % Bottom right footer
\renewcommand\headrulewidth{0.4pt} % Size of the header rule
\renewcommand\footrulewidth{0.4pt} % Size of the footer rule

\setlength\parindent{0pt} % Removes all indentation from paragraphs

%----------------------------------------------------------------------------------------
%	CODE INCLUSION CONFIGURATION
%----------------------------------------------------------------------------------------

\definecolor{pblue}{rgb}{0.13,0.13,1}
\definecolor{pgreen}{rgb}{0,0.5,0}
\definecolor{pred}{rgb}{0.9,0,0}
\definecolor{pgrey}{rgb}{0.46,0.45,0.48}

\lstset{language=Java,
  showspaces=false,
  showtabs=false,
  breaklines=true,
  showstringspaces=false,
  breakatwhitespace=true,
  commentstyle=\color{pgreen},
  keywordstyle=\color{pblue},
  stringstyle=\color{pred},
  basicstyle=\ttfamily,
  moredelim=[il][\textcolor{pgrey}]{$$},
  moredelim=[is][\textcolor{pgrey}]{\%\%}{\%\%}
}
% Creates a new command to include a perl script, the first parameter is the filename of the script (without .pl), the second parameter is the caption
\newcommand{\cfile}[2]{
\begin{itemize}
\item[]\lstinputlisting[caption=#2,label=#1]{#1.c}
\end{itemize}
}

%----------------------------------------------------------------------------------------
%	DOCUMENT STRUCTURE COMMANDS
%	Skip this unless you know what you're doing
%----------------------------------------------------------------------------------------

% Header and footer for when a page split occurs within a problem environment
\newcommand{\enterProblemHeader}[1]{
\nobreak\extramarks{#1}{#1 continued on next page\ldots}\nobreak
\nobreak\extramarks{#1 (continued)}{#1 continued on next page\ldots}\nobreak
}

% Header and footer for when a page split occurs between problem environments
\newcommand{\exitProblemHeader}[1]{
\nobreak\extramarks{#1 (continued)}{#1 continued on next page\ldots}\nobreak
\nobreak\extramarks{#1}{}\nobreak
}

\setcounter{secnumdepth}{0} % Removes default section numbers
\newcounter{homeworkProblemCounter} % Creates a counter to keep track of the number of problems

\newcommand{\homeworkProblemName}{}
\newenvironment{homeworkProblem}[1][Exercício \arabic{homeworkProblemCounter}]{ % Makes a new environment called homeworkProblem which takes 1 argument (custom name) but the default is "Problem #"
\stepcounter{homeworkProblemCounter} % Increase counter for number of problems
\renewcommand{\homeworkProblemName}{#1} % Assign \homeworkProblemName the name of the problem
\section{\homeworkProblemName} % Make a section in the document with the custom problem count
\enterProblemHeader{\homeworkProblemName} % Header and footer within the environment
}{
\exitProblemHeader{\homeworkProblemName} % Header and footer after the environment
}

\newcommand{\problemAnswer}[1]{ % Defines the problem answer command with the content as the only argument
\noindent\framebox[\columnwidth][c]{\begin{minipage}{0.98\columnwidth}#1\end{minipage}} % Makes the box around the problem answer and puts the content inside
}

\newcommand{\homeworkSectionName}{}
\newenvironment{homeworkSection}[1]{ % New environment for sections within homework problems, takes 1 argument - the name of the section
\renewcommand{\homeworkSectionName}{#1} % Assign \homeworkSectionName to the name of the section from the environment argument
\subsection{\homeworkSectionName} % Make a subsection with the custom name of the subsection
\enterProblemHeader{\homeworkProblemName\ [\homeworkSectionName]} % Header and footer within the environment
}{
\enterProblemHeader{\homeworkProblemName} % Header and footer after the environment
}

%----------------------------------------------------------------------------------------
%	NAME AND CLASS SECTION
%----------------------------------------------------------------------------------------

\newcommand{\hmwkTitle}{Prática 1} % Assignment title
\newcommand{\hmwkDueDate}{Quarta-feira,\ 5 de\ Abril de\ 2017} % Due date
\newcommand{\hmwkClass}{BC1513} % Course/class
\newcommand{\hmwkClassTime}{} % Class/lecture time
\newcommand{\hmwkClassInstructor}{João H. Kleinschmidt} % Teacher/lecturer
\newcommand{\hmwkAuthorName}{Rodrigo Martins de Oliveira} % Your name

%----------------------------------------------------------------------------------------
%	TITLE PAGE
%----------------------------------------------------------------------------------------

\title{
\vspace{2in}
\textmd{\textbf{\hmwkClass:\ \hmwkTitle}}\\
\normalsize\vspace{0.1in}\small{Para\ \hmwkDueDate}\\
\vspace{0.1in}\large{\textit{\hmwkClassInstructor\ \hmwkClassTime}}
\vspace{3in}
}

\author{\textbf{\hmwkAuthorName}}
\date{} % Insert date here if you want it to appear below your name

%----------------------------------------------------------------------------------------

\begin{document}

\maketitle

%----------------------------------------------------------------------------------------
%	TABLE OF CONTENTS
%----------------------------------------------------------------------------------------

%\setcounter{tocdepth}{1} % Uncomment this line if you don't want subsections listed in the ToC

%\newpage
%\tableofcontents
\newpage

\section{Introdução}

Nesta prática de laboratório de redes de computadores são exercitados os conceitos de comunicação entre aplicações através do protocolo de transporte não-orientado à conexão UDP. Os conceitos de endereço IP, porta e socket são cobertos.

\section{Desenvolvimento}

%----------------------------------------------------------------------------------------
%	PROBLEM 1
%----------------------------------------------------------------------------------------

% To have just one problem per page, simply put a \clearpage after each problem
\begin{homeworkProblem}
%Listing \ref{11009713_desafio2} shows a the radixSort\_base() function and auxiliary functions and macros used.
%
%\cfile{11009713_desafio2}{Threading a Binary Tree in ERD order}

\section{a)}
O servidor UDP abre um socket UDP em uma porta específica e ``escuta'' pacotes que chegam por esta porta, capturando-os e lendo informações nele contidas. Após receber um pacote o servidor capitaliza o conteúdo deste e o envia de volta para o endereço e porta de origem informados pelo pacote.

O cliente UDP abre um socket UDP em uma porta qualquer e envia um pacote para o endereço e porta esperados do servidor e espera pela resposta na mesma porta.

\section{b)}
Como não há quaisquer mecanismos mais complexos de verificação do status de rede do servidor e de confirmações de envio e recebimento de pacotes, o pacote enviado pelo cliente é perdido já que o servidor ainda não está online e escutando a porta designada, então o cliente fica ``eternamente'' aguardando a resposta do servidor. Mesmo que o servidor venha a ficar online posteriormente, o pacote enviado anteriormente pelo cliente já foi perdido e o servidor não saberá disso, portanto, continuará aguardando a chegada de um novo pacote.

\section{c)}
Se a porta para a qual o cliente envia os pacotes não for a mesma porta a qual o servidor está conectado qualquer pacote enviado pelo cliente será perdido (ou capturado por outra aplicação que detém controle sobre a porta) e o servidor nunca receberá os pacotes. O resultado é semelhante ao que foi visto no exercício 1.b)

\end{homeworkProblem}

%----------------------------------------------------------------------------------------
%	PROBLEM 2
%----------------------------------------------------------------------------------------

% To have just one problem per page, simply put a \clearpage after each problem
\begin{homeworkProblem}
%Listing \ref{11009713_desafio2} shows a the radixSort\_base() function and auxiliary functions and macros used.
%
%\cfile{11009713_desafio2}{Threading a Binary Tree in ERD order}

Desde que o endereço de IP e porta do servidor estejam corretamente configurados no cliente, a comunicação entre os dois processos acontece normalmente.

\end{homeworkProblem}

%----------------------------------------------------------------------------------------
%	PROBLEM 3
%----------------------------------------------------------------------------------------

% To have just one problem per page, simply put a \clearpage after each problem
\begin{homeworkProblem}
%Listing \ref{11009713_desafio2} shows a the radixSort\_base() function and auxiliary functions and macros used.
%
%\cfile{11009713_desafio2}{Threading a Binary Tree in ERD order}

Sim, o servidor receberá todas as mensagens. Pelo fato de ser uma comunicação não-orientada à conexão, todos os pacotes, de diferentes origens, são enfileirados pela camada de transporte e enviados sequencialmente para o servidor lê-los, portanto a concorrência de recebimento de múltiplos pacotes de diferentes endereços de origem é abstraída da aplicação.

\end{homeworkProblem}

%----------------------------------------------------------------------------------------
%	PROBLEM 4
%----------------------------------------------------------------------------------------

% To have just one problem per page, simply put a \clearpage after each problem
\begin{homeworkProblem}
%Listing \ref{11009713_desafio2} shows a the radixSort\_base() function and auxiliary functions and macros used.
%
%\cfile{11009713_desafio2}{Threading a Binary Tree in ERD order}

O servidor não precisa sofrer quaisquer alterações dada sua abstração sobre a origem dos pacotes, para o servidor, o endereço e porta de origem do pacote apenas são um parâmetro e não definem qualquer tratamento de sessão especial (como seria no caso de uma conexão TCP, em que uma conexão é estabelecida).

\begin{lstlisting}
public class UDPClient {
    public static void main(String[] args) throws SocketException,
            UnknownHostException, IOException
    {
        BufferedReader inFromUser =
                new BufferedReader(new InputStreamReader(System.in));
        DatagramSocket clientSocket = new DatagramSocket();
        InetAddress IPAddress = InetAddress.getByName("127.0.0.1");
        byte[] sendData = new byte[1024];
        byte[] receiveData = new byte[1024];
        String sentence;
        while (true) {
            sentence = inFromUser.readLine();
            if ("sair".equals(sentence)) {
                break;
            }
            sendData = sentence.getBytes();
            DatagramPacket sendPacket = new DatagramPacket(sendData, sendData.length, IPAddress, 9876);
            clientSocket.send(sendPacket);
            DatagramPacket receivePacket = new DatagramPacket(receiveData, receiveData.length);
            clientSocket.receive(receivePacket);
            String modifiedSentence = new String(receivePacket.getData());
            System.out.println("Do servidor:" + modifiedSentence);
        }
        clientSocket.close();
    }
}
\end{lstlisting}

\end{homeworkProblem}

%----------------------------------------------------------------------------------------
%	PROBLEM 5
%----------------------------------------------------------------------------------------

% To have just one problem per page, simply put a \clearpage after each problem
\begin{homeworkProblem}
%Listing \ref{11009713_desafio2} shows a the radixSort\_base() function and auxiliary functions and macros used.
%
%\cfile{11009713_desafio2}{Threading a Binary Tree in ERD order}

\begin{lstlisting}
public class WhatsappServer {
    public static void main(String[] args) throws SocketException, IOException {
        DatagramSocket serverSocket = new DatagramSocket(9876);
        byte[] receiveData = new byte[1024];
        byte[] sendData = new byte[1024];
        List<String> blacklist = Arrays.asList("172.31.33.27", "172.31.33.29");
        while(true)
        {
            DatagramPacket receivePacket
                    = new DatagramPacket(receiveData, receiveData.length);
            System.out.println("Servidor aguardando..." );
            serverSocket.receive(receivePacket);
            InetAddress IPAddress = receivePacket.getAddress();
            
            if (blacklist.contains(IPAddress.toString())) {
                continue;
            }
                    
            String sentence = new String( receivePacket.getData());
            System.out.println("Mensagem recebida: " + sentence);
            int port = receivePacket.getPort();
            String capitalizedSentence = sentence.toUpperCase();
            sendData = capitalizedSentence.getBytes();
            DatagramPacket sendPacket
                    = new DatagramPacket(sendData, sendData.length, IPAddress, port);
            serverSocket.send(sendPacket);
        }
    }
}
\end{lstlisting}

\end{homeworkProblem}

%----------------------------------------------------------------------------------------
%	PROBLEM 5
%----------------------------------------------------------------------------------------

% To have just one problem per page, simply put a \clearpage after each problem
\begin{homeworkProblem}
%Listing \ref{11009713_desafio2} shows a the radixSort\_base() function and auxiliary functions and macros used.
%
%\cfile{11009713_desafio2}{Threading a Binary Tree in ERD order}

patients.xds

\end{homeworkProblem}

%----------------------------------------------------------------------------------------
%	PROBLEM 7
%----------------------------------------------------------------------------------------

% To have just one problem per page, simply put a \clearpage after each problem
\begin{homeworkProblem}
%Listing \ref{11009713_desafio2} shows a the radixSort\_base() function and auxiliary functions and macros used.
%
%\cfile{11009713_desafio2}{Threading a Binary Tree in ERD order}

A aplicação do Whatsapp foi criada contendo: (i) um servidor principal que é responsável por receber as mensagens de clientes e garantir que sejam recebidas aos respectivos clientes de destino; e (ii) aplicações cliente que enviam mensagens endereçadas a outros clientes para o servidor e recebem mensagens do servidor remetidas por outros clientes.

Cada aplicação cliente possui um identificador único obtido informado pelo servidor e pode enviar mensagens endereçando-as a outros clientes através de seus respectivos identificadores. O cliente espera receber informações sobre o recebimento da mensagem pelo destinatário e também sobre mensagens recebidas.

O servidor espera que cada cliente envie mensagens formatadas segundo um determinado padrão e envia respostas de confirmação de recebimento para os clientes e espera que eles também enviem respostas de confirmação de recebimento de mensagens para poder informar os remetentes das mensagens sobre sua entrega.

Cada cliente possui um mecanismo de \emph{retry} para o envio de mensagens para garantir que elas sejam entregues ao servidor.

O servidor guarda representações de conversas entre clientes, nas quais enfileira mensagens a serem entregues. Quando as mensagens são entregues aos destinatários o servidor apaga suas cópias locais.

\end{homeworkProblem}

%----------------------------------------------------------------------------------------

\section{Conclusão}

Nesta prática observamos que o protocolo UDP permite abstrair a origem dos pacotes de dados, já que não requer conexão, permitindo ao servidor lidar fácilmente com múltiplos clientes quando o tratamento \emph{per} pacote é isolado e independente.

Também é notável que o protoloco UDP requer cuidados especiais para garantir o recebimento de pacotes.

\end{document}